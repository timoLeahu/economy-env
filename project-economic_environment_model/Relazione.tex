% Per compilarlo, cliccate su Menu, selezionate come compilatore "pdfLaTeX", TeX Live Version 2020 (o uno degli altri, whatever works), e come documento principale "Relazione.tex"

%%% Preambolo del documento %%%
\documentclass[a4paper, titlepage]{article}
\usepackage[T1]{fontenc}
\usepackage[utf8]{inputenc}
\usepackage[english]{babel}
\usepackage{amsmath}
\usepackage{listings}
\usepackage{textcomp}
\usepackage{multirow}
\usepackage{multicol}
\usepackage{booktabs}
\usepackage{graphicx}
\usepackage{floatflt}
\usepackage{epsfig}
\usepackage{pstricks}
\usepackage{subfigure}
\usepackage[labelfont=bf, font=scriptsize]{caption}
\usepackage[english]{varioref}
\usepackage[suftesi]{frontespizio}
\usepackage{color}
\usepackage{tikz}
\usepackage{caption}
\usepackage{pgfplots}
\usepackage{comment}
\usepackage{lipsum}
\pgfplotsset{compat=1.16}
%% my add
\usepackage{enumitem}


\begin{document}

%%% Frontespizio %%%
\begin{frontespizio}
\Universita{Padova} % CTT
\Logo{Figure/logo_unipd} % CTT
\Divisione{Dipartimento di Ing. Civile, Edile e Ambientale} % CTT
\Corso[Laurea]{Mathematical Engineering} % CTT, a meno che non cambi la denominazione del corso
\Annoaccademico{2023-24}
\Titoletto{Project in Dynamical Systems} % CTT
\Titolo{Economic growth and \\Environmental pollution}
\Sottotitolo{November 2023}
\NCandidati{Gruppo di lavoro} % CTT
\Candidato[2039113]{Timofei Leahu}
\NRelatore{Docente}{} % CTT
\Relatore{Prof. Antonio Ponno} % CTT, a meno che non sia cambiato il Prof.
\end{frontespizio}
\IfFileExists{\jobname-frn.pdf}{}{\immediate\write18{pdflatex \jobname-frn}} % ASSOLUTAMENTE CTT, è il comando che materialmente vi genera il frontespizio.

%%% Indice %%%
\tableofcontents

\newpage
%%% Sezioni %%%
% Qui inizia la relazione vera e propria.
% Le Sezioni sono singoli file .tex dentro la cartella Sezioni. Potete a libera scelta scrivere tutto su un singolo file e chiamare all'interno di questi con il comando \section{} le varie sezioni, oppure dividere le singole sezioni in più file Sezione_i.tex, e poi inserirle tutte con \input{Sezioni/Sezione_i.tex} per i = 1 ... N
% Titolo della sezione e label. Vi consiglio, per questioni di ordine mentale e rapidità successiva di reference, di etichettare le label in modo sensato, con riferimento chiaro a cosa si sta etichettando. Quindi sec:nomesezione per una sezione, im:nomeimmagine per una immagine, e via dicendo.
\section{Introduction}\label{sec:introduction} 
Understanding the causes behind the huge differences in standards of living across countries has been a central issue in economics since the time of classical economists. Economic growth is an issue that, as Robert Lucas (1988, p. 5, \cite{lucas_mechanics_1988}) points out: “Once one starts to think about [economic growth], it is hard to think about anything else.” 
Traditionally economists were concentrated in the growth theory and regularization in the growth process, and not much attention has been given to the relationship between economic growth and the environment until recent decades. To quote a statement from \cite{xepapadeas_chapter_2005}, “Received growth theory is biased. It neglects to take into account
the pollution costs of economic growth.” Therefore, in the last decades, several research has been undertaken which tries to explore the links between economic growth and the environment, especially regarding issues associated with the impact of natural resources on growth processes and sustainability. Consequently, in the last years, growth theory has taken into account also the interrelationships between environment pollution, capital accumulations and the growth of variables which are of central importance in that theory. There are some evidence about growth that economists have long taken for granted, and they are based of five basic facts (see Paul Romer 1994, p. 12, \cite{romer_origins_1994}): 
\begin{enumerate}
    \item There are many firms in a market economy.
    \item Discoveries differ from inputs in the sense that many people can use them at the
    same time.
    \item It is possible to replicate physical activities.
    \item Technological advance comes from things that people do.
    \item Many individuals and firms have market power and earn monopoly rents from discoveries.
\end{enumerate}
If the environmental dimension is to be incorporated into the main body of growth
theory, then a sixth fact should be added:
\begin{enumerate}[resume]
	\item There is joint production of a flow of waste material that degrades the environment, and environmental quality is positively valued by individuals.
\end{enumerate}
The purpose is therefore to explore how fact six is incorporated into
modern growth theory, and it concentrates on the relationship between economic growth and environmental pollution.

The evolution of growth theory since the 1950s has passed through two main stages. The basic feature of the first stage, which originated with the Solow model (see \cite[Solow]{solow_contribution_1956}, \cite[Swan]{swan_economic_1956}), is that technical change is exogenous. This means that growth rates cannot be affected by the government policy. In this stage, growth is analyzed either in terms of models with exogenous saving rates (the Solow–Swan model), or models where consumption and hence savings are determined by optimizing individuals. These are the so-called optimal growth or \textbf{Ramsey models} (\cite[Ramsey]{ramsey_mathematical_1928}, \cite[Cass]{cass_optimum_1965}, \cite[Koopmans]{koopmans_concept_1963}). The main feature of the second stage that emerged in the 1980s is that technical change is endogenized in such a way that economic growth is associated with an endogenous outcome of the economic system rather than with exogenous forces. In the context of endogenous growth models, growth rates can be affected by government policies. The purpose with this work is to explore vital questions such as: 
\begin{itemize}
	\item is environmental protection compatible with economic growth;
	\item is it possible to have sustained growth in the long run without accumulation of pollution;
	\item what is the impact of environmental concerns on growth, and in particular, how are the levels, the paths or the growth rates of crucial variables such as capital, income, consumption or environmental pollution affected if we take into account the environment;
\end{itemize}
>> TODO... 
Must add the structure of the article here...

Nonetheless, this work is based also on the paper \cite{caravaggio_nonlinear_2018} with focus on the model 2.2 [The Ramsey-Cass-Koopmans Model With Environmental Pollution]. 



	
\section{The environmental pollution model}\label{Sec:model}
The model will unify the process of economic growth with the environment, an economic model describing technology and preferences which characterize the economic problem should be linked to the environmental module which describes the natural process characterizing pollution accumulation. Relation between economic and environmental module is motivated by the following facts:
\begin{itemize}
	\item Environmental pollution is a by-product of production or consumption processes
	taking place in the economic module.
	\item Emissions generated in the economic module affect the flow or the accumulation
	of pollutants in the ambient environment (e.g., emissions of sulphur oxides, noise,
	carbon dioxide accumulation in the atmosphere, or phosphorus accumulation in
	water bodies).
	\item Environmental pollution has detrimental effects on the utility of individuals.
	\item Environmental pollution could have detrimental productivity effects, while improvements in environmental quality might have productivity enhancing effects.
\end{itemize}
Consequently, given a neoclassical aggregate production function for the economy, 
\begin{equation}\label{eq:prod-func}
	Y = F(K, AL), 
\end{equation} 
where $K$ is the capital stock and $AL$ is the effective labour\footnote{ In the context of a Solow model, if labour time is denoted by L and labour's effectiveness, or knowledge, is A, then by effective labour they mean AL. In general means 'efficiency units' of labour or 'productive effort' as opposed to time spent.}, to allow for labour augmenting technical change, the flow of emissions at time t can be written as
\begin{equation}\label{eq:flow-emiss}
	Z(t) = v(Y(t)). 
\end{equation}
One can also specify \eqref{eq:flow-emiss} as $Z=\phi Y$, where $\phi$ is the unit emission coefficient, that is, emissions per unit of output. Emissions reducing technologies can be incorporated by further specifying the unit emission coefficient as $\phi (K)$, with $\phi' (K)<0$ for $K\in \mathcal{K} \subset \mathcal{R_+}$. From this assumption it is clear that if the capital stock accumulates, new "cleaner" techniques are used. 
\paragraph{Remark.} The stock of capital can be also split into productive capital, which is the pollution generating capital $K_p$, and abatement capital $K_a$, which is the pollution reducing capital. In this case, the production function can be written as 
\begin{equation}\label{eq:prod-func-capital-split}
	Y = F(K_p, AL, K_a)
\end{equation}
and the flow of emissions can be specified as $Z=\phi (K_a)Y$. In the next sections the capital will not be split for simplicity. \\

Another formulation \cite{brock_polluted_1973} allows for the flow of pollution to be an input in the production function:
\begin{equation}\label{eq:prod-funct-with-poll}
	Y = F(K, AL, BZ), 
\end{equation}
where $BZ$ is the effective flow of pollution as an input, for input augmenting technical change.\footnote{The idea behind this formulation is that “techniques of production are less costly in terms of capital inputs if more pollution is allowed” (\cite{brock_polluted_1973}, p. 443).}

Another way of modeling the environment into production is to consider that environmental quality, $E$, is a factor of production. This formulation captures productivity effects of the environment such as health of workers and the production function can be written as 
\begin{equation}
	Y = F(K, AL, E), \quad\text{with}\quad \frac{\partial Y}{\partial E}>0.
\end{equation}

Damages from environmental pollution can be associated with either the flow of emissions per unit time, such as smoke or noise, or the stock of pollution as emissions are accumulated in the ambient environment, such as greenhouse gases, or heavy metals. When the stock of pollution, denoted by $P$, is of interest, then its accumulation is usually represented by a transition equation:
\begin{equation}\label{eq:poll-dyn-nonLin}
	\dot{P} = Z - mP + h(P),
\end{equation}
where $m$ reflects exponential pollution decay and $h(P)$ represents a nonlinear feedback, sometimes called internal loadings. The term $h(P)$ in \eqref{eq:poll-dyn-nonLin} is relatively important because could give to the scientist a more precise information about what is going on in the real situation. Most of the time the use of the only linear dynamics, as implied by the exponential pollution decay, to model natural processes might not be a good approximation and, hence, a nonlinear structure induced by feedbacks might be more appropriate. In general, the $h(P)$ function is assumed to be S-shaped, and a common functional form used in applications is $h(P)=P^2 /(1+P^2)$.\footnote{Feedbacks could be positive if the impact is such that the initial perturbation is enhanced, or negative if the initial perturbation is reduced. For example, in the study of climate change, a positive feedback exists when an increase in temperature – say due to increased accumulation of greenhouse gases – increases evaporation from the oceans, which brings more water vapor into the atmosphere and finally enhances greenhouse effects.}
Ignoring these nonlinearities might obscure very important characteristics that we observe in reality, such as bifurcations of a natural system to alternative equilibrium states, irreversibilities or hysteresis, which could be important in exploring the true nature of the relationship between growth and the environment.

The evolution of environmental quality can be described by a formulation which is equivalent to modeling environmental quality as a renewable resource, or 
\begin{equation}\label{eq:dyn-env-quality}
	\dot{E}=R(E)-Z,
\end{equation}
where $R(E)$ is an environmental regeneration function and Z represents reduction in environmental quality, or natural capital, from the flow of emissions, through an extractive-like process.
Either \eqref{eq:poll-dyn-nonLin} or \eqref{eq:dyn-env-quality} can be used to describe the state of the environment.

The environmental dimension is introduced into the utility function by defining a function which includes both consumption and environmental quality among the factors determining the satisfaction derived by individuals. Environmental quality appears as disutility from pollution. Thus we have for the $i$-th individual:
$$
	U(c_i, Z) \quad\text{or}\quad U(c_i, P),
$$
depending on whether the flow or stock of pollution affects utility. In a decentralized economy, individuals treat environmental quality as fixed when maximizing their utility.

When discussing social optimization, the criterion function for the government or the social planner takes the form of a \textit{felicity functional}\footnote{This follows terminology introduced by Arrow and Kurz \cite{kruz_public_1970}.} with additive utilities over time and identical individuals
$$
\int_{0}^{\infty} e^{-\rho t} N(t)U(\bar{c}(t), P(t)) \,dt,
$$
where $N(t)$ is the population at time t, $\bar{c}$ is per capita consumption and $\rho \geq 0$ represents the discount rate for future utilities, or rate of time preference.\footnote{For a more detailed analysis of the foundations of a welfare criterion that incorporates environmental concerns, see \cite{heal_chapter_2005}. Further properties of this functional will also be presented in Section \ref{Sec:opt-grow-env-poll} of this chapter.} A similar functional could be written for the case where the utility function includes $Z$ instead of $P$.

If the analysis is carried out in terms of the stock of environmental quality, then the above functional should be written as \footnote{Most of the analysis in the rest of this work will use mainly the formulation where pollution causes disutility as a public bad. The use of the stock of environmental quality as a utility and productivity enhancing stock leads to approximately equivalent results.}
$$
\int_{0}^{\infty} e^{-\rho t} N(t)U(\bar{c}(t), E(t)) \,dt.
$$



\section{Conclusion}\label{Sec:conclusion}
Write down the conclusions here!!!

\clearpage
%%% Bibliography %%%
\addcontentsline{toc}{section}{References}
\bibliographystyle{plain}
\bibliography{biblio}

\newpage
%%% Appendici %%%
% Eventuali appendici vanno qui. Se non avete appendici da inserire, togliete/commentate queste righe.
%\appendix
%\section{Appendice A}\label{app:Appendice A}
\lipsum	

\end{document}
