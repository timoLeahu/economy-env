% Per compilarlo, cliccate su Menu, selezionate come compilatore "pdfLaTeX", TeX Live Version 2020 (o uno degli altri, whatever works), e come documento principale "Relazione.tex"

%%% Preambolo del documento %%%
\documentclass[a4paper, titlepage]{article}
\usepackage[T1]{fontenc}
\usepackage[utf8]{inputenc}
\usepackage[english]{babel}
\usepackage{amsmath}
\usepackage{listings}
\usepackage{textcomp}
\usepackage{multirow}
\usepackage{multicol}
\usepackage{booktabs}
\usepackage{graphicx}
\usepackage{floatflt}
\usepackage{epsfig}
\usepackage{pstricks}
\usepackage{subfigure}
\usepackage[labelfont=bf, font=scriptsize]{caption}
\usepackage[english]{varioref}
\usepackage[suftesi]{frontespizio}
\usepackage{color}
\usepackage{tikz}
\usepackage{caption}
\usepackage{pgfplots}
\usepackage{comment}
\usepackage{lipsum}
\pgfplotsset{compat=1.16}
%% my add
\usepackage{enumitem}


\begin{document}

%%% Frontespizio %%%
\begin{frontespizio}
\Universita{Padova} % CTT
\Logo{Figure/logo_unipd} % CTT
\Divisione{Dipartimento di Ing. Civile, Edile e Ambientale} % CTT
\Corso[Laurea]{Mathematical Engineering} % CTT, a meno che non cambi la denominazione del corso
\Annoaccademico{2023-24}
\Titoletto{Project in Dynamical Systems} % CTT
\Titolo{Economic growth and \\Environmental pollution}
\Sottotitolo{November 2023}
\NCandidati{Gruppo di lavoro} % CTT
\Candidato[2039113]{Timofei Leahu}
\NRelatore{Docente}{} % CTT
\Relatore{Prof. Antonio Ponno} % CTT, a meno che non sia cambiato il Prof.
\end{frontespizio}
\IfFileExists{\jobname-frn.pdf}{}{\immediate\write18{pdflatex \jobname-frn}} % ASSOLUTAMENTE CTT, è il comando che materialmente vi genera il frontespizio.

%%% Indice %%%
\tableofcontents

\newpage
%%% Sezioni %%%
% Qui inizia la relazione vera e propria.
% Le Sezioni sono singoli file .tex dentro la cartella Sezioni. Potete a libera scelta scrivere tutto su un singolo file e chiamare all'interno di questi con il comando \section{} le varie sezioni, oppure dividere le singole sezioni in più file Sezione_i.tex, e poi inserirle tutte con \input{Sezioni/Sezione_i.tex} per i = 1 ... N
% Titolo della sezione e label. Vi consiglio, per questioni di ordine mentale e rapidità successiva di reference, di etichettare le label in modo sensato, con riferimento chiaro a cosa si sta etichettando. Quindi sec:nomesezione per una sezione, im:nomeimmagine per una immagine, e via dicendo.
\section{Introduction}\label{sec:introduction} 
Understanding the causes behind the huge differences in standards of living across countries has been a central issue in economics since the time of classical economists. Economic growth is an issue that, as Robert Lucas (1988, p. 5, cited in \cite{maler_chapter_2005}) points out: “Once one starts to think about [economic growth], it is hard to think about anything else.” 
Traditionally economists were concentrated in the growth theory and regularization in the growth process, and not much attention has been given to the relationship between economic growth and the environment until recent decades. To quote a statement from \cite{maler_chapter_2005}, “Received growth theory is biased. It neglects to take into account
the pollution costs of economic growth.” Therefore, in the last decades, several research has been undertaken which tries to explore the links between economic growth and the environment, especially regarding issues associated with the impact of natural resources on growth processes and sustainability. Consequently, in the last years, growth theory has taken into account also the interrelationships between environment pollution, capital accumulations and the growth of variables which are of central importance in that theory. There are some evidence about growth that economists have long taken for granted, and they are based of five basic facts (see Paul Romer 1994, p. 12, \cite{romer_origins_1994}): 
\begin{enumerate}
    \item There are many firms in a market economy.
    \item Discoveries differ from inputs in the sense that many people can use them at the
    same time.
    \item It is possible to replicate physical activities.
    \item Technological advance comes from things that people do.
    \item Many individuals and firms have market power and earn monopoly rents from discoveries.
\end{enumerate}
If the environmental dimension is to be incorporated into the main body of growth
theory, then a sixth fact should be added:
\begin{enumerate}[resume]
	\item There is joint production of a flow of waste material that degrades the environment, and environmental quality is positively valued by individuals.
\end{enumerate}
The purpose is therefore to explore how fact six is incorporated into
modern growth theory, and it concentrates on the relationship between economic growth and environmental pollution.

The evolution of growth theory since the 1950s has passed through two main stages. The basic feature of the first stage, which originated with the Solow model (see \cite[Solow]{solow_contribution_1956}, \cite[Swan]{swan_economic_1956}), is that technical change is exogenous. This means that growth rates cannot be affected by the government policy. In this stage, growth is analyzed either in terms of models with exogenous saving rates (the Solow–Swan model), or models where consumption and hence savings are determined by optimizing individuals. These are the so-called optimal growth or \textbf{Ramsey models} (\cite[Ramsey]{ramsey_mathematical_1928}, \cite[Cass]{cass_optimum_1965}, \cite[Koopmans]{koopmans_concept_1963}). The main feature of the second stage that emerged in the 1980s is that technical change is endogenized in such a way that economic growth is associated with an endogenous
outcome of the economic system rather than with exogenous forces. In the context of endogenous growth models, growth rates can be affected by government policies.


Nonetheless, this work is based also on the paper \cite{caravaggio_nonlinear_2018} with focus on the model 2.2 [The Ramsey-Cass-Koopmans Model With Environmental Pollution]. 
This is test writing in order to see if the the git is sensitive!! <<<

\subsection{Esempio di figura}\label{subsec:esempio_figura}
Se volete inserire una imagine/figura, trovate l'esempio qui sotto.
\begin{figure}[h]
    \centering
    \includegraphics[width=0.5\textwidth]{Figure/esempio.jpg}
    \caption{Esempio di figura. La caption di una Figura va sempre sotto la figura.}
    \label{fig:esempio_fig}
\end{figure}\\
Il che genera la Fig.\,\ref{fig:esempio_fig}.\\
Formati che piacciono a LaTeX: in pratica tutti, ma solitamente si trovano i soliti .pdf, .eps e .jpg
\cite[Chap 12]{Liptser-S-77}

\subsection{Esempio di tabella}\label{subsec:esempio_tabella}
Se invece volete inserire una tabella, l'esempio lo trovate qui sotto, in Tab.\,\ref{tab:esempio_tab}

\begin{table}[ht]
    \centering
    \caption{Esempio di Tabella. La caption di una Tabella va sempre sopra.}
    \label{tab:esempio_tab}
    \begin{tabular}{cl}
    \toprule
    COLA & COLB   \\
    \midrule
    A & D   \\
    B & E   \\
    C & F   \\
    \bottomrule
    \end{tabular}
\end{table}

\begin{table}[ht]
\centering
    \caption{Esempio di Tabella multirow.}
    \label{tab:esempio_tab_multirow}
    \begin{tabular}{cc}
    \toprule
    \multirow{2}{*}{Multirow}&X\\
    &X\\
    \bottomrule
    \end{tabular}
\end{table}

\begin{table}[!ht]
\centering
    \caption{Esempio di Tabella multicolumn.}
    \label{tab:esempio_tab_multicolumn}
    \begin{tabular}{cc}
    \toprule
    \multicolumn{2}{c}{Multi-column}\\
    \midrule
    X&X\\
    \bottomrule
    \end{tabular}
\end{table}
	
\section{The environmental pollution model}\label{Sec:model}
The model will unify the process of economic growth with the environment, an economic model describing technology and preferences which characterize the economic problem should be linked to the environmental module which describes the natural process characterizing pollution accumulation. Relation between economic and environmental module is motivated by the following facts:
\begin{itemize}
	\item Environmental pollution is a by-product of production or consumption processes
	taking place in the economic module.
	\item Emissions generated in the economic module affect the flow or the accumulation
	of pollutants in the ambient environment (e.g., emissions of sulphur oxides, noise,
	carbon dioxide accumulation in the atmosphere, or phosphorus accumulation in
	water bodies).
	\item Environmental pollution has detrimental effects on the utility of individuals.
	\item Environmental pollution could have detrimental productivity effects, while improvements in environmental quality might have productivity enhancing effects.
\end{itemize}
Consequently, given a neoclassical aggregate production function for the economy, 
\begin{equation}\label{eq:prod-func}
	Y = F(K, AL), 
\end{equation} 
where $K$ is the capital stock and $AL$ is the effective labour\footnote{ In the context of a Solow model, if labour time is denoted by L and labour's effectiveness, or knowledge, is A, then by effective labor one means AL. In general means 'efficiency units' of labour or 'productive effort' as opposed to time spent.}, to allow for labour augmenting technical change, the flow of emissions at time t can be written as
\begin{equation}\label{eq:flow-emiss}
	Z(t) = v(Y(t)). 
\end{equation}
 One can also specify \eqref{eq:flow-emiss} as $Z=\phi Y$, where $\phi$ is the unit emission coefficient, that is, emissions per unit of output. Emissions reducing technologies can be incorporated by further specifying the unit emission coefficient as $\phi (K)$, with $\phi' (K)<0$ for $K\in \mathcal{K} \subset \mathcal{R_+}$. From this assumption it is clear that if the capital stock accumulates, new "cleaner" techniques are used. 
 
 
\paragraph{Esempio di equazione.} %Paragrafo
Ok, quindi avete visto come si fa per generare una tabella o inserire una figura. E per quanto riguarda le Equazioni?\\
Semplice, nel modo seguente
\begin{equation}\label{eq:EinsteinField_eq}
R_{\mu \nu} - \dfrac{1}{2} R g_{\mu \nu} + \Lambda g_{\mu \mu} = \dfrac{8 \pi G}{c^4} T_{\mu \nu}
\end{equation}
E la si richiama nel testo con il solito comando {\it ref}, quindi l'Equazione di Campo di cui sopra è l'Eq.\,\ref{eq:EinsteinField_eq}.\\
Se ho necessità di scrivere più formule consecutivamente, si usa eqnarray:
\begin{eqnarray}
\dfrac{\partial \mathcal D}{\partial t} &=& \nabla \times \mathcal H \\
\dfrac{\partial \mathcal B}{\partial t} &=& -\nabla \times \mathcal E \\
\nabla \cdot \mathcal B  &=& 0 \\
\nabla \cdot \mathcal D  &=& 0 
\end{eqnarray}
E se invece di quattro le voglio considerare come fosse una sola:
\begin{eqnarray}\label{eq:Maxwell_eq}
\dfrac{\partial \mathcal D}{\partial t} &=& \nabla \times \mathcal H \nonumber \\
\dfrac{\partial \mathcal B}{\partial t} &=& -\nabla \times \mathcal E \nonumber \\
\nabla \cdot \mathcal B  &=& 0 \nonumber \\
\nabla \cdot \mathcal D  &=& 0 
\end{eqnarray}


\section{Conclusion}\label{Sec:conclusion}
Write down the conclusions here!!!

\clearpage
%%% Bibliography %%%
\addcontentsline{toc}{section}{References}
\bibliographystyle{plain}
\bibliography{biblio}

\newpage
%%% Appendici %%%
% Eventuali appendici vanno qui. Se non avete appendici da inserire, togliete/commentate queste righe.
%\appendix
%\input{Sezioni/Appendice.tex}	

\end{document}
