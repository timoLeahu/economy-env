\section{The environmental pollution model}\label{Sec:model}
The model will unify the process of economic growth with the environment, an economic model describing technology and preferences which characterize the economic problem should be linked to the environmental module which describes the natural process characterizing pollution accumulation. Relation between economic and environmental module is motivated by the following facts:
\begin{itemize}
	\item Environmental pollution is a by-product of production or consumption processes
	taking place in the economic module.
	\item Emissions generated in the economic module affect the flow or the accumulation
	of pollutants in the ambient environment (e.g., emissions of sulphur oxides, noise,
	carbon dioxide accumulation in the atmosphere, or phosphorus accumulation in
	water bodies).
	\item Environmental pollution has detrimental effects on the utility of individuals.
	\item Environmental pollution could have detrimental productivity effects, while improvements in environmental quality might have productivity enhancing effects.
\end{itemize}
Consequently, given a neoclassical aggregate production function for the economy, 
\begin{equation}\label{eq:prod-func}
	Y = F(K, AL), 
\end{equation} 
where $K$ is the capital stock and $AL$ is the effective labour\footnote{ In the context of a Solow model, if labour time is denoted by L and labour's effectiveness, or knowledge, is A, then by effective labor one means AL. In general means 'efficiency units' of labour or 'productive effort' as opposed to time spent.}, to allow for labour augmenting technical change, the flow of emissions at time t can be written as
\begin{equation}\label{eq:flow-emiss}
	Z(t) = v(Y(t)). 
\end{equation}
 One can also specify \eqref{eq:flow-emiss} as $Z=\phi Y$, where $\phi$ is the unit emission coefficient, that is, emissions per unit of output. Emissions reducing technologies can be incorporated by further specifying the unit emission coefficient as $\phi (K)$, with $\phi' (K)<0$ for $K\in \mathcal{K} \subset \mathcal{R_+}$. From this assumption it is clear that if the capital stock accumulates, new "cleaner" techniques are used. 
 
 
\paragraph{Esempio di equazione.} %Paragrafo
Ok, quindi avete visto come si fa per generare una tabella o inserire una figura. E per quanto riguarda le Equazioni?\\
Semplice, nel modo seguente
\begin{equation}\label{eq:EinsteinField_eq}
R_{\mu \nu} - \dfrac{1}{2} R g_{\mu \nu} + \Lambda g_{\mu \mu} = \dfrac{8 \pi G}{c^4} T_{\mu \nu}
\end{equation}
E la si richiama nel testo con il solito comando {\it ref}, quindi l'Equazione di Campo di cui sopra è l'Eq.\,\ref{eq:EinsteinField_eq}.\\
Se ho necessità di scrivere più formule consecutivamente, si usa eqnarray:
\begin{eqnarray}
\dfrac{\partial \mathcal D}{\partial t} &=& \nabla \times \mathcal H \\
\dfrac{\partial \mathcal B}{\partial t} &=& -\nabla \times \mathcal E \\
\nabla \cdot \mathcal B  &=& 0 \\
\nabla \cdot \mathcal D  &=& 0 
\end{eqnarray}
E se invece di quattro le voglio considerare come fosse una sola:
\begin{eqnarray}\label{eq:Maxwell_eq}
\dfrac{\partial \mathcal D}{\partial t} &=& \nabla \times \mathcal H \nonumber \\
\dfrac{\partial \mathcal B}{\partial t} &=& -\nabla \times \mathcal E \nonumber \\
\nabla \cdot \mathcal B  &=& 0 \nonumber \\
\nabla \cdot \mathcal D  &=& 0 
\end{eqnarray}

