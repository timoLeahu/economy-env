% Titolo della sezione e label. Vi consiglio, per questioni di ordine mentale e rapidità successiva di reference, di etichettare le label in modo sensato, con riferimento chiaro a cosa si sta etichettando. Quindi sec:nomesezione per una sezione, im:nomeimmagine per una immagine, e via dicendo.
\section{Introduction}\label{sec:introduction} 
Understanding the causes behind the huge differences in standards of living across countries has been a central issue in economics since the time of classical economists. Economic growth is an issue that, as Robert Lucas (1988, p. 5, \cite{lucas_mechanics_1988}) points out: “Once one starts to think about [economic growth], it is hard to think about anything else.” 
Traditionally economists were concentrated in the growth theory and regularization in the growth process, and not much attention has been given to the relationship between economic growth and the environment until recent decades. To quote a statement from \cite{xepapadeas_chapter_2005}, “Received growth theory is biased. It neglects to take into account
the pollution costs of economic growth.” Therefore, in the last decades, several research has been undertaken which tries to explore the links between economic growth and the environment, especially regarding issues associated with the impact of natural resources on growth processes and sustainability. Consequently, in the last years, growth theory has taken into account also the interrelationships between environment pollution, capital accumulations and the growth of variables which are of central importance in that theory. There are some evidence about growth that economists have long taken for granted, and they are based of five basic facts (see Paul Romer 1994, p. 12, \cite{romer_origins_1994}): 
\begin{enumerate}
    \item There are many firms in a market economy.
    \item Discoveries differ from inputs in the sense that many people can use them at the
    same time.
    \item It is possible to replicate physical activities.
    \item Technological advance comes from things that people do.
    \item Many individuals and firms have market power and earn monopoly rents from discoveries.
\end{enumerate}
If the environmental dimension is to be incorporated into the main body of growth
theory, then a sixth fact should be added:
\begin{enumerate}[resume]
	\item There is joint production of a flow of waste material that degrades the environment, and environmental quality is positively valued by individuals.
\end{enumerate}
The purpose is therefore to explore how fact six is incorporated into
modern growth theory, and it concentrates on the relationship between economic growth and environmental pollution.

The evolution of growth theory since the 1950s has passed through two main stages. The basic feature of the first stage, which originated with the Solow model (see \cite[Solow]{solow_contribution_1956}, \cite[Swan]{swan_economic_1956}), is that technical change is exogenous. This means that growth rates cannot be affected by the government policy. In this stage, growth is analyzed either in terms of models with exogenous saving rates (the Solow–Swan model), or models where consumption and hence savings are determined by optimizing individuals. These are the so-called optimal growth or \textbf{Ramsey models} (\cite[Ramsey]{ramsey_mathematical_1928}, \cite[Cass]{cass_optimum_1965}, \cite[Koopmans]{koopmans_concept_1963}). The main feature of the second stage that emerged in the 1980s is that technical change is endogenized in such a way that economic growth is associated with an endogenous outcome of the economic system rather than with exogenous forces. In the context of endogenous growth models, growth rates can be affected by government policies. The purpose with this work is to explore vital questions such as: 
\begin{itemize}
	\item is environmental protection compatible with economic growth;
	\item is it possible to have sustained growth in the long run without accumulation of pollution;
	\item what is the impact of environmental concerns on growth, and in particular, how are the levels, the paths or the growth rates of crucial variables such as capital, income, consumption or environmental pollution affected if we take into account the environment;
\end{itemize}
>> TODO... 
Must add the structure of the article here...

Nonetheless, this work is based also on the paper \cite{caravaggio_nonlinear_2018} with focus on the model 2.2 [The Ramsey-Cass-Koopmans Model With Environmental Pollution]. 



